\documentclass{article}


% if you need to pass options to natbib, use, e.g.:
%     \PassOptionsToPackage{numbers, compress}{natbib}
% before loading neurips_2024


% ready for submission
\usepackage{neurips_2024}


% to compile a preprint version, e.g., for submission to arXiv, add add the
% [preprint] option:
%     \usepackage[preprint]{neurips_2024}


% to compile a camera-ready version, add the [final] option, e.g.:
%     \usepackage[final]{neurips_2024}


% to avoid loading the natbib package, add option nonatbib:
%    \usepackage[nonatbib]{neurips_2024}


\usepackage[utf8]{inputenc} % allow utf-8 input
\usepackage[T1]{fontenc}    % use 8-bit T1 fonts
\usepackage{hyperref}       % hyperlinks
\usepackage{url}            % simple URL typesetting
\usepackage{booktabs}       % professional-quality tables
\usepackage{amsfonts}       % blackboard math symbols
\usepackage{nicefrac}       % compact symbols for 1/2, etc.
\usepackage{microtype}      % microtypography
\usepackage{xcolor}         % colors


\title{Unveiling the Syntax Within: Interpreting Grammar Embeddings in Meta’s LLaMA Models}


% The \author macro works with any number of authors. There are two commands
% used to separate the names and addresses of multiple authors: \And and \AND.
%
% Using \And between authors leaves it to LaTeX to determine where to break the
% lines. Using \AND forces a line break at that point. So, if LaTeX puts 3 of 4
% authors names on the first line, and the last on the second line, try using
% \AND instead of \And before the third author name.


\author{%
  Pratim Chowdhary\thanks{Use footnote for providing further information
    about author (webpage, alternative address)---\emph{not} for acknowledging
    funding agencies.} \\
  Department of Computer Science\\
  Dartmouth College\\
  \texttt{cpratim.25@dartmouth.edu} \\
  % examples of more authors
  \And
  Peter Chin \\
  Department of Engineering\\
  Thayer School of Engineering\\
  \texttt{pc@dartmouth.edu} \\
  \And
  Deepernab Chakrabarty \\
  Department of Computer Science\\
  Dartmouth College\\
  \texttt{deepernab@dartmouth.edu} \\
  % \AND
  % Coauthor \\
  % Affiliation \\
  % Address \\
  % \texttt{email} \\
  % \And
  % Coauthor \\
  % Affiliation \\
  % Address \\
  % \texttt{email} \\
  % \And
  % Coauthor \\
  % Affiliation \\
  % Address \\
  % \texttt{email} \\
}


\begin{document}


\maketitle


\begin{abstract}
  This paper investigates the mechanisms by which large language models (LLMs) encode 
  grammatical knowledge, focusing on Meta's LLaMA models. By leveraging embedding vectors, 
  we classify grammatically correct sentences and analyze the activation patterns of 
  attention heads to identify their roles in processing specific grammatical structures. 
  Furthermore, we explore the effects of selectively removing these attention heads, shedding 
  light on how grammar is embedded within the model's architecture. Our findings aim to enhance
  the understanding of LLMs' linguistic capabilities and their internal organization of syntactic knowledge.
\end{abstract}


\section{Introduction}

\section{Related Work}

\begin{enumerate}

  \item \textbf{Deciphering Stereotypes in Pre-Trained Language Models} 
  \begin{itemize}
    \item This paper goes into details about how stereotypes are encoded in the embeddings of pre-trained language models and how specific attention heads are responsible for encoding them.
  \end{itemize}
  \item \textbf{}

\end{enumerate}


\section{Methodology}


\section{Experiments}

\section{Discussion}

\section{Conclusion}



\section*{References}


References follow the acknowledgments in the camera-ready paper. Use unnumbered first-level heading for
the references. Any choice of citation style is acceptable as long as you are
consistent. It is permissible to reduce the font size to \verb+small+ (9 point)
when listing the references.
Note that the Reference section does not count towards the page limit.
\medskip


{
\small


[1] Alexander, J.A.\ \& Mozer, M.C.\ (1995) Template-based algorithms for
connectionist rule extraction. In G.\ Tesauro, D.S.\ Touretzky and T.K.\ Leen
(eds.), {\it Advances in Neural Information Processing Systems 7},
pp.\ 609--616. Cambridge, MA: MIT Press.


[2] Bower, J.M.\ \& Beeman, D.\ (1995) {\it The Book of GENESIS: Exploring
  Realistic Neural Models with the GEneral NEural SImulation System.}  New York:
TELOS/Springer--Verlag.


[3] Hasselmo, M.E., Schnell, E.\ \& Barkai, E.\ (1995) Dynamics of learning and
recall at excitatory recurrent synapses and cholinergic modulation in rat
hippocampal region CA3. {\it Journal of Neuroscience} {\bf 15}(7):5249-5262.
}

%%%%%%%%%%%%%%%%%%%%%%%%%%%%%%%%%%%%%%%%%%%%%%%%%%%%%%%%%%%%

\appendix

\section{Appendix / supplemental material}


Optionally include supplemental material (complete proofs, additional experiments and plots) in appendix.

\end{document}